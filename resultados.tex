\newpage
\clearpage

\section{Resultados e Discussões}
\label{result:result}

- Classificação de Imagens
        - Apresentar processo de preservação espacial, lime, fine tunning e custo computacional.
        - Será que o uso de transferência de aprendizado foi bom nesse caso? Pois as camadas de pooling não têm parâmetros, mas a atuação das mesmas acabam influenciando as demais camadas.
        - A imagens menores (CIFAR 100) possuem muito menos informações em relação às de maior resolução (Food 101), então quando a aplicação do pooling ocorre, as imagens perdem todo o significado semântico.
        - As camadas mais profundas, responsáveis pelas informações de mais baixo nível, incluindo linhas e bordas que dão senso de espacialidade, sofrem maior impacto de acurácia do que camadas mais superficiais.
        - O método proposto se apresenta pior em ambos os datasets.

- Segmentação de Imagens com U-net e U-net-like
        - Não há uma discrepância significativa em relação às métricas quando comparado com max pooling, mas se apresenta mais lento e ainda tem menores métricas.
        - Valeria um teste com outros datasets contendo objetos de diversos tamanhos, como lost and found, cityscape ou celulas. Mas a adaptação de código com o framework tensforflow se torna custosa, valeria a migração para o pytorch tentando minimizar essa barreira tecnológica.
        - Valeria a implementação de um método de unpooling com o processo reverso de bpca (colocar a fórmula e imagens), a diferença é que esse método contaria com parâmetros, sendo necessário usar uma camada densa por baixo dos panos.
        - O treinamento com U-Net-Like realmente ocorrem mais rápido do que as U-Nets convencionais, visto que a quantidade de parâmetros é menor e as normalizações de batch normalization trazem o beneficio das otimizações.
        - Na prática em relação não tem muita mudança no resultado final.

\subsection{Considerações Finais do Capítulo}
\label{result:final}
