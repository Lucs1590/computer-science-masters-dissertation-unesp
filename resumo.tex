\section*{Resumo}
\thispagestyle{empty}
\makeatletter
\newcommand*{\rom}[1]{\expandafter\@slowromancap\romannumeral #1@}
\makeatother
Em meio ao campo da visão computacional, atividades relacionadas à segmentação de imagens têm proporcionado avanços quanto a análises médicas mais acuradas, entendimento de cenas, projetos de sistemas autônomos, entre outros estudos semelhantes, os quais têm ganho amplitude devido ao advento das redes neurais artificiais e das técnicas de aprendizado profundo, que proporcionam base para o desenvolvimento de muitos modelos e arquiteturas que almejam alcançar o estado-da-arte, proporcionando melhores desempenhos para as questões de segmentação de imagens.
% Redes Convolucionais, camada de pooling e BPCA
Destarte, o presente trabalho propõe apresentar uma visão geral sobre conceitos de redes neurais artificiais e contextualizar técnicas de segmentações tradicionais e modernas, apresentando suas vantagens e desvantagens mediante a algumas problemáticas, alvitrando uma nova técnica para as camadas de \textit{pooling} de modelos baseados em segmentação semântica, de efeito a contribuir com fatores de espacialidade no processo de segmentação dos objetos presente na entrada das redes.
Finalmente, vale citar que ...