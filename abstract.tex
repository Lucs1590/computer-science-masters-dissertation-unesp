\section*{Abstract}
\thispagestyle{empty}
Computer vision, especially in the field of image segmentation, has made significant contributions to advanced medical analysis, understanding complex scenes, design of autonomous systems, among other areas of possible applications. The emergence of deep learning techniques has enabled the advancement of architectures and models with a view to achieving the state-of-the-art in image segmentation.
However, there are still significant obstacles to be overcome, particularly the simultaneous preservation of spatiality and maintaining a global view of the attribute map during the dimensionality reduction procedure inherent in the pooling layers of convolutional networks.
Faced with this challenge, this work proposes the investigation and development of the Block-based Principal Component Analysis Pooling (BPCAPooling) method. Different from the conventional Max Pooling method, BPCAPooling is a PCA-based pooling method that seeks not only to locally preserve the spatial information of samples, but also to maintain a global view of attribute maps while reducing dimensionality.
To evaluate the effectiveness of this method, it was first applied to the architecture of convolutional neural networks such as the Visual Geometry Group (VGG) with 16 weight layers, and later extended to more complex architectures with the challenge of image segmentation, namely U-Nets. and its variants.
The experiments showed that, although the application of BPCAPooling in classification models did not outperform traditional methods in terms of metrics such as accuracy and loss, visual differences were observed when compared with conventional pooling methods . Furthermore, in the semantic segmentation task, the method proved to be a viable alternative, reaching a score of 0.3333 in Mean Intersection over Union (mIoU), accuracy of 86.77\% and loss of 0.6659 .

\textbf{Keywords:} Computer vision; Image segmentation; Convolutional neural networks; Preservation of spatiality; Block-based Principal Component Analysis Pooling (BPCAPooling).