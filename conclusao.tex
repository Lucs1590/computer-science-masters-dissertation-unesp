\newpage
\clearpage
\section{Conclusões}
\label{conclusion}
As técnicas de segmentação representam uma ferramenta valiosa para a identificação de regiões de interesse em imagens, abordando desde situações mais simples, abordadas pelas técnicas de segmentação tradicionais (Capítulo \ref{segment}), até cenários mais complexos, resolvidos pela segmentação semântica (Capítulo \ref{semantic}). Este último tipo de segmentação, em particular, apresenta semelhanças com a compreensão humana de cenários \citep{Mohan2020}, uma vez que classifica cada pixel em uma cena, atribuindo-os a diversas classes. Tal capacidade só foi alcançada graças à evolução das arquiteturas de redes neurais profundas (Capítulo \ref{deep}), que apresentam habilidades melhoradas de aprendizado e adaptação em resposta aos problemas fornecidos.

Em uma análise mais específica, focada no campo do aprendizado de imagens, as redes neurais convolucionais ganharam destaque (Capítulo \ref{cnn}). Esta evolução entre as redes neurais profundas, as convolucionais e a segmentação semântica direcionou o estudo para o teste de um método de \textit{pooling}, introduzido no contexto das CNNs, que preservasse a espacialidade das amostras, mantendo seus valores fundamentais, como explorado no Capítulo \ref{project}. O desenvolvimento desse método respondeu à primeira questão levantada nos objetivos deste trabalho, delineados na introdução (\ref{intro}).

A aplicação do método \textit{Block-based Principal Component Analysis Pooling}, ou BPCAPooling, acompanhou a evolução das redes, iniciando sua aplicação em uma arquitetura comumente utilizada para as CNNs, a VGG-16. O objetivo era coletar insumos e avaliar a sua aplicabilidade como substituto dos métodos convencionais. Os resultados e discussões desses experimentos, detalhados no Capítulo \ref{results}, revelaram que os métodos convencionais - \textit{Max Pooling} e \textit{Avg Pooling} - ainda são superiores nos dois conjuntos de dados testados.

O melhor modelo utilizando \textit{Max Pooling} alcançou uma acurácia de $46,071\%$ e uma \textit{loss} de $1,4862$, no primeiro conjunto de dados, enquanto obteve uma acurácia de $50,080\%$ e uma \textit{loss} de $2,3745$ no segundo conjunto. Já o uso do BPCAPooling resultou em apenas $17,543\%$ de acurácia e $3,5121\%$ de \textit{loss} para o primeiro conjunto, e $4,4130\%$ de acurácia e $4,4404$ de \textit{loss} para o segundo conjunto. Portanto, os resultados indicam que o método desenvolvido não é recomendado como uma alternativa viável para substituir os métodos convencionais.

Apesar dos valores de acurácia e \textit{loss} relativamente baixos, o método apresentou diferenças visuais consideráveis em relação ao método tradicional. Essas diferenças foram evidentes ao utilizar ferramentas de explicabilidade de modelos e ao examinar características das saídas das redes, conforme mencionado na Seção \ref{results:class:datasets}.

No contexto das segmentações semânticas, os resultados obtidos (Capítulo \ref{results}) revelaram comportamento diferente dos casos de classificação ao aplicar o método proposto nas arquiteturas. Os modelos U-Net com o método proposto apresentaram uma mIoU de $0,3333$, acurácia de $86,77\%$ e uma \textit{loss} de $0,6659$, enquanto o modelo com uso de \textit{Max Pooling} obteve uma mIoU de $0,3367$ e acurácia de $88,61\%$, mas com uma \textit{loss} ligeiramente superior ($0,6668$). No entanto, todos os resultados mostram diferenças mínimas tanto em métricas quanto em qualidade visual das segmentações, sugerindo o BPCAPooling como uma alternativa válida ao método convencional, não fosse pela complexidade de código agregada, o que aumenta significativamente o tempo de execução durante as fases de treinamento e validação.

Portanto, para futuros desenvolvimentos, sugere-se: 1) a otimização do código desenvolvido, considerando a troca de \textit{frameworks}, visando a redução da complexidade do código; 2) testes em conjuntos de dados adicionais apropriados para a tarefa de segmentação semântica; e 3) o desenvolvimento do método BPCAUnpooling (como sugerido na Seção \ref{results:semantic:future}) para preservar a espacialidade das amostras na fase de decodificação das U-Nets.

Por fim, todo o código fonte utilizado neste trabalho está documentado e disponível como \textit{open-source}, buscando fomentar contribuições e incentivar a comunidade científica a avançar no tema abordado, facilitando futuras pesquisas nesta área a partir de um ponto inicial avançado e com algumas dificuldades mitigadas.
