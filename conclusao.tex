\newpage
\clearpage
\section{Conclusões}
\label{final:final}

Dentre as segmentações disponíveis e estudadas no decorrer desses estudos, entre as segmentações tradicionais (Capítulo \ref{segment:image}), segmentações semânticas (Capítulo \ref{semantic:semantic}), segmentações de instâncias (Capítulo \ref{instance:instance}) e segmentações panópticas (Capítulo \ref{panoptic:panoptic}), ficou evidente que a mais próxima do olhar e compreensão humana, que é atingido desde a infância \cite{Mohan2020}, é a segmentação panóptica, visto que classifica todos os pixels de uma cena, além de proporcionar a separação de instâncias, podendo ter uma distinção dentre os objetos de mesma classe.

Em relação à parte odontológica, constata-se que os métodos utilizados para segmentação concentram-se em técnicas tradicionais \cite{Hammad2020}, tal como o uso de \textit{k-means}, ou técnicas baseadas em modelos 3D, algo que ficou claro com base na Seção \ref{proposal:revision}. Todavia, é válido dizer que redes preparadas para segmentação panóptica ainda não foram aplicadas em imagens odontológicas fotográficas, suscitando oportunidades para o desenvolvimento e contribuição com os campos de segmentação em âmbito odontológico, dos quais se cita a contagem de instâncias de dentes e a atribuição hierárquica entre os componentes odontológicos.

Quanto às segmentações panópticas, é valido dizer que essas são totalmente adequadas para contextos que tenham uma problemática em que todos os pixels de determinada cena possuem relevância e são determinados como região de interesse, o que se aplica para as situações de entendimento de cenas, como as odontológicas, onde todo pixel em uma boca, por exemplo, pode ser classificado como uma classe e as instâncias também são claramente separadas, de modo que, com o advento da técnica de segmentação panóptica \textit{part-aware} \cite{DeGeus2021}, tornou-se possível segmentar as partes das classes, como as cáries dentro de um dente, em um processo único, que já identifica os dentes e as cáries em cada um deles.

Todavia, mesmo com a disponibilidade de toda essa tecnologia, ainda é possível observar pontos desfavoráveis para o desenvolvimento, como a pequena variação de conjuntos de dados que estejam preparados e anotados para o uso em segmentações panópticas com partes, sendo que os disponíveis estão normalmente relacionados a cidades e instâncias pertencentes à mesmas classes, além das métricas de qualidades baixas alcançadas pelos modelos de segmentação panóptica observados nas Tabelas \ref{conclusion:table:1} e \ref{conclusion:table:2}, o que abre oportunidades para a modificação das redes base, como é proposto pelo presente trabalho, com a expectativa de melhoria de técnicas de segmentações panópticas, assim como melhorias relacionadas a indicadores de $PQ$ e $Part-PQ$.

Por fim, em resumo, por meio do presente trabalho, anseia-se conquistar a segmentação hierárquica de componentes visuais odontológicos ao determinar um arcabouço que utiliza a tecnologia \textit{panoptic segmentation part-aware}, bem como alterações nos modelos base de segmentação, alcançando métricas mais altas em relação aos indicadores atuais e tendo, como consequência, um aumento de desempenho quanto às análises de dentistas, com a redução do tempo de análise.
