\section*{Resumo}
\thispagestyle{empty}
\makeatletter
\makeatother
Em meio ao campo da visão computacional, atividades relacionadas à segmentação de imagens têm proporcionado avanços quanto a análises médicas mais acuradas, entendimento de cenas, projetos de sistemas autônomos, entre outros estudos semelhantes, os quais têm ganhado amplitude devido ao advento das redes neurais artificiais e das técnicas de aprendizado profundo, que proporcionam base para o desenvolvimento de muitos modelos e arquiteturas que almejam alcançar o estado-da-arte, proporcionando melhores desempenhos para as questões de segmentação de imagens.

Contudo, apesar dos avanços, o desafio da preservação da espacialidade durante a redução de dimensionalidade, particularmente nas camadas de \textit{pooling} das redes convolucionais, continua sendo um ponto crucial. Este trabalho se concentra na investigação e desenvolvimento de um método para preservar a espacialidade das amostras durante a redução de dimensionalidade. Nesse contexto, é proposto o método denominado \textit{Block-based Principal Component Analysis Pooling} (BPCAPooling), um método de \textit{pooling} baseado em PCA que visa conservar a estrutura espacial das amostras, permitindo uma representação mais precisa das características aprendidas para as próximas camadas da rede.

Este estudo aborda a aplicação do BPCAPooling em arquiteturas de redes neurais convolucionais para a tarefa de classificação, em específico as VGG-16, como forma de obter resultados inicias em relação ao desempenho do método e, posteriormente, em arquiteturas com o desafio de segmentação de imagens, em específico as U-Nets e suas variações. Os resultados obtidos indicam que o BPCAPooling apresente diferenças visuais em comparação com os métodos convencionais de \textit{pooling} e, embora os os modelos desenvolvidos com o BPCAPooling não superem os métodos tradicionais em termos de métricas de desempenho para as classificações, como acurácia e \textit{loss}, apresentam-se como alternativas consideráveis na tarefa de segmentação semântica, chegando a $0,3333$ de mIoU, $86,77\%$ de acurácia e $0,6659$ de \textit{loss}.

\textbf{Palavras-chave:} Visão computacional; Segmentação de imagens; Redes neurais convolucionais; Preservação de espacialidade; \textit{Block-based Principal Component Analysis Pooling} (BPCAPooling).
