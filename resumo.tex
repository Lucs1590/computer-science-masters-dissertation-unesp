\section*{Resumo}
\thispagestyle{empty}
\makeatletter
\makeatother
A visão computacional, especialmente no âmbito de segmentação de imagens, tem promovido contribuições significativas para análises médicas avançadas, compreensão de cenas complexas, design de sistemas autônomos, entre outros domínios de possíveis aplicações. A emergência de técnicas de aprendizado profundo tem possibilitado o avanço de arquiteturas e modelos com vistas a alcançar primeiramente o estado-da-arte na segmentação de imagens.
No entanto, ainda há obstáculos significativos a serem superados, particularmente a preservação simultânea da espacialidade e a manutenção de uma visão global do mapa de atributos durante o procedimento de redução de dimensionalidade inerente às camadas de \textit{pooling} das redes convolucionais.
Frente a esse desafio, este trabalho propõe a investigação e o desenvolvimento do método Block-based Principal Component Analysis Pooling (BPCAPooling). Diferente do método convencional \textit{Max Pooling}, o BPCAPooling é um método de pooling baseado no PCA que busca não apenas preservar a espacialidade das amostras, mas também manter uma visão global dos mapas de atributos enquanto reduz a dimensionalidade.
Para avaliar a eficácia deste método, foi aplicado primeiramente na arquitetura de redes neurais convolucionais como a Visual Geometry Group (VGG) com 16 camadas de peso, e posteriormente estendido a arquiteturas mais complexas com o desafio de segmentação de imagens, nomeadamente as U-Nets e suas variantes.
Os experimentos mostraram que, embora a aplicação do BPCAPooling em modelos de classificação não tenha superado os métodos tradicionais em termos de métricas como acurácia e \textit{loss}, observaram-se diferenças visuais na comparação com os métodos de \textit{pooling} convencionais. Além disso, na tarefa de segmentação semântica, o método mostrou-se como uma alternativa viável, atingindo uma pontuação de 0,3333 em Mean Intersection over Union (mIoU), acurácia de 86,77\% e \textit{loss} de 0.6659.

\textbf{Palavras-chave:} Visão computacional; Segmentação de imagens; Redes neurais convolucionais; Preservação de espacialidade; Block-based Principal Component Analysis Pooling (BPCAPooling).