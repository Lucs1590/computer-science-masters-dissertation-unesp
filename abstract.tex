\section*{Abstract}
\thispagestyle{empty}
In the field of computer vision, activities related to image segmentation have provided advances in more accurate medical analysis, scene understanding, autonomous systems projects, among other similar studies, which have gained breadth due to the advent of artificial neural networks and deep learning techniques, which provide the basis for the development of many models and architectures that aim to reach the state-of-the-art, providing better performance for image segmentation issues.
However, despite advances, the challenge of preserving spatiality during dimensionality reduction, particularly in the pooling layers of convolutional networks, remains a crucial point. This work focuses on the investigation and development of a method to preserve the spatiality of samples during dimensionality reduction. In this context, the method called Block-based Principal Component Analysis Pooling (BPCAPooling) is proposed, a pooling method based on PCA that aims to conserve the spatial structure of the samples, allowing a more accurate representation of the characteristics learned for the next layers of the network.
This study addresses the application of BPCAPooling in convolutional neural network architectures for the classification task, specifically VGG-16, as a way of obtaining initial results in relation to the method's performance and, later, in architectures with the challenge of data segmentation. images, specifically U-Nets and their variations. The results obtained indicate that BPCAPooling presents visual differences compared to conventional pooling methods and, although the models developed with BPCAPooling do not outperform traditional methods in terms of performance metrics for classifications, such as accuracy and loss, present themselves as considerable alternatives in the task of semantic segmentation, reaching $0.3333$ in mIoU, $86.77\%$ in accuracy and $0.6659$ in loss.

\textbf{Keywords:} Computer vision; Image segmentation; Convolutional neural networks; Preservation of spatiality; \textit{Block-based Principal Component Analysis Pooling} (BPCAPooling).